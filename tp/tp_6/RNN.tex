\section{Code d'un RNN}\label{RNN-code}
\begin{lstlisting}[language=C, caption={Constants for a simple RNN}]
/**
* @brief This code illustrate a small RNN
* @author M. DERRAZ
*/
#include <Arduino.h>

// Constants
const int inputSize = 5;       // Number of input neurons
const int hiddenSize = 10;     // Number of hidden neurons
const int outputSize = 1;      // Number of output neurons
const int sequenceLength = 20; // Length of the sequence to be predicted
const int ledPin = 13;         // Pin for the LED display
// Variables
float inputs[inputSize];                 // Input sequence
float outputs[outputSize];               // Output sequence
float hidden[hiddenSize];                // Hidden state
float weightsIH[hiddenSize][inputSize];  // Input-hidden weights
float weightsHH[hiddenSize][hiddenSize]; // Hidden-hidden weights
float weightsHO[outputSize][hiddenSize]; // Hidden-output weights
float biasH[hiddenSize];                 // Hidden biases
float biasO[outputSize];                 // Output biases
float threshold = 0.5;                   // Prediction threshold
bool predict = false;                    // Flag for starting prediction sequence

/// Funtions declaration
void initializeWeights();
void forwardPass(float *input, float *output, float *hidden);
float activation(float x);
void readPoten();
void startPrediction();
\end{lstlisting}

\newpage

\begin{lstlisting}[language=C, caption={Loop and Setup functions for a simple RNN}]
void setup()
{
    // Setup pins
    Serial.begin(115200);
    pinMode(ledPin, OUTPUT);
    // Initialize weights and biases
    initializeWeights();
}

void loop()
{
    // Read potentiometer value and adjust threshold
    readPoten();
    // Check if start button is pressed

    startPrediction();

    if (predict)
    {
    forwardPass(inputs, outputs, hidden);

    // Output predicted value on LED display
    if (outputs[0] > threshold)
    {
        digitalWrite(ledPin, HIGH);
    }
    else
    {
        digitalWrite(ledPin, LOW);
    }

    Serial.print("Output: ");
    Serial.println(outputs[0]);

    // Shift input sequence
    for (int i = sequenceLength - 1; i > 0; i--)
    {
        inputs[i] = inputs[i - 1];
    }
    inputs[0] = outputs[0];
    }
}
\end{lstlisting}

\newpage

\begin{lstlisting}[language=C, caption={RNN Functions to activate, predict and train the NN}]
/**
* @brief Initialize weights and biases
*/
void initializeWeights()
{
    // Initialize input-hidden weights
    for (int i = 0; i < hiddenSize; i++)
    {
    for (int j = 0; j < inputSize; j++)
    {
        weightsIH[i][j] = random(-100, 100) / 100.0;
    }
    }
    // Initialize hidden-hidden weights
    for (int i = 0; i < hiddenSize; i++)
    {
    for (int j = 0; j < hiddenSize; j++)
    {
        weightsHH[i][j] = random(-100, 100) / 100.0;
    }
    }
    // Initialize hidden-output weights
    for (int i = 0; i < outputSize; i++)
    {
    for (int j = 0; j < hiddenSize; j++)
    {
        weightsHO[i][j] = random(-100, 100) / 100.0;
    }
    }
    // Initialize biases
    for (int i = 0; i < hiddenSize; i++)
    {
    biasH[i] = random(-100, 100) / 100.0;
    }
    for (int i = 0; i < outputSize; i++)
    {
    biasO[i] = random(-100, 100) / 100.0;
    }
}

/**
* @brief Forward pass of the RNN to compute the output sequence
*/
void forwardPass(float *input, float *output, float *hidden)
{
    // Compute hidden state
    for (int i = 0; i < hiddenSize; i++)
    {
    hidden[i] = 0;
    for (int j = 0; j < inputSize; j++)
    {
        hidden[i] += weightsIH[i][j] * input[j];
    }
    for (int j = 0; j < hiddenSize; j++)
    {
        hidden[i] += weightsHH[i][j] * hidden[j];
    }
    hidden[i] += biasH[i];
    hidden[i] = activation(hidden[i]);
    }
    // Compute output
    for (int i = 0; i < outputSize; i++)
    {
    output[i] = 0;
    for (int j = 0; j < hiddenSize; j++)
    {
        output[i] += weightsHO[i][j] * hidden[j];
    }
    output[i] += biasO[i];
    output[i] = activation(output[i]);
    }
}

/**
* @brief Sigmoid activation function
*/
float activation(float x)
{
    // Sigmoid activation function
    return 1.0 / (1.0 + exp(-x));
}

/**
* @brief Read potentiometer value and adjust prediction threshold
*/
void readPoten()
{
    // Read poten value and map to prediction threshold
    int potValue = 0.5; // exemple
}


/**
* @brief Start prediction sequence by initializing input sequence with random values
* and by setting the predict flag to true
*/
void startPrediction()
{
    // Initialize input sequence with random values
    for (int i = 0; i < sequenceLength; i++)
    {
    inputs[i] = random(0, 10) / 10.0;
    }
    // Set flag to start prediction sequence
    predict = true;
}  
\end{lstlisting}